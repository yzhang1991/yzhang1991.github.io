\documentstyle[times,11pt,verbatim,js-singlespace]{article}  % 10pt default?

\title{COSC6340: Database Systems\\
Mini-DBMS Project
}

\author{Instructor: Carlos Ordonez}
\date{}

\begin{document}

\maketitle

\section{Introduction}

In this project you will develop a small DBMS in C++. 
Your program will be focused on managing SPJ queries in
large amounts of data. As a result, your small DBMS
will require you to use binary files.
You will have to submit this program in two phases.
The initial phase is focused on storage and 
selection of data (SELECT and JOIN).
the second phase is based on sorting and 
grouping techniques (GROUP BY and ORDER BY).
Each submission will be graded on the specific aspects of the related phase.
However, you are expected to fix the bugs or missing functionalities
from previous phases in order to finish future phases.

\section{Program requirements}

Your program should be coded in C++ using
the Borland C++ v. 5.5 Compiler.
The program needs to efficiently manage
large amounts of data. Therefore, your application
cannot use regular text files to store the
data. Instead, you are required to use binary files.

The first phase of the project requires you to manage two modes
of entering SQL queries: via an SQL console and using
a batch file. Your SQL console should start if the user decides
to give as, an argument, an SQL query, and once it has finished executing, allow
the user to input more SQL queries by displaying a prompt (``$SQL>$").
The application will finalize when the user types the ``quit;" command.

Additionally, your initial phase of the project requires
a catalog table that contains all the metadata information
for all the tables. This catalog table is going to be
maintained and managed in main memory, and updated
when the application closes (lazy update).
The metadata of a table includes the table names, the column names (and their types),
primary keys, size of the record, table and number of records.
All temporary tables should be included in the memory catalog table, but not in the
text catalog.
%This temporary tables should be included in the ``memory" catalog, 
%but do not store them back when 
%update is taking place.

Some other commands that are required for the initial part of the project are
the following.
``SHOW TABLE", which will return the DDL information of the specified table.
CREATE TABLE, which will create a table in the DBMS. INSERT INTO should
add given record to the required specified table. The SELECT/FROM
operator should return the specified columns without duplicates.
The SELECT/FROM/WHERE query will be similar to the SELECT/FROM operation but,
also considers the comparison criterion given in the where clause ($<,>,=$). For the JOIN
operation, you are only required to cross two tables with
only a single attribute as a join condition. If you have repeated table names, they should
include the source table name (e.g. T1\_A , T2\_A ).
INSERT INTO would insert all the data of a SELECT operation in the specified table. 
If you are trying to insert data into a table that does not exist you should
return an error.

For the second part of the project, you are required to implement an ORDER BY operation
of a specified table (using an efficient algorithm like merge sort). 
In addition, a new operation called INSERT INTO/ORDER BY will
be implemented, with the requirement that it should store the data retrieved by the SELECT
in an ordered fashion in the specified table.
Finally, a SELECT/GROUP BY operation will be performed, where
you are required to exploit your sorting algorithm from the previous
operation to obtain this functionality. You can assume that
only a SUM aggregation will be performed on numerical data.

You will be given +2 points (on a 10 scale) if you can include (AND,OR and NOT) operations in the WHERE clause for the
first phase. Also, you will be given +1 point (second phase) if you can include an indexing technique by using
any library of your choice that includes B+Trees or Hash indexing.


\subsection{Programming details}

\subsubsection{Phase 1}

In the initial phase, you are required to
call your application as ``uhdbms.exe", from which you can
send an SQL query or
a batch file containing a set of SQL queries.
Your application should be able to store and select data
in an efficient manner.
Therefore, you are required to
store each table in a separate
binary file. Each binary file will
have the name of the table and a ``.tbl" extension.
The data types that you should manage include:
integers (4 bytes) and chars (1 byte per char).
Every time you complete an operation, you are expected
to return the result/data of such operation (e.g. Created TABLE T Successfully).

In order to efficiently manage all your database collection and data, you
are required to have a file containing all the information of your database schema (table metadata).
This catalog should be maintained (updated) with a lazy policy. As a result, your
text file will only be updated with the required information
when the program finishes executing.
Hence, the metadata of each table should
be included in text file (catalog.txt) with a 
table descriptor per line.
The table metadata includes:
tablename, columns, primary key, record size in bytes,
total number of bytes of the table,
and the total number of records in the table.
The columns should be divided by ``," and
the type of the column should be divided by a ``:".
At the beginning of your program, you can load
the whole catalog in memory, and only modify this
memory loaded catalog. Once your program finishes executing, you
can then dump the new catalog information 
(discarding the information regarding temporary tables).
Due to the fact that you are expected to keep your temporary tables during the execution
of the program in the catalog, you also have to keep the related binary files and
allow the user the capability of also querying (SELECT) this temporary tables.
%variable length fields divided by ``$|$".
%The metadata of each table should be included
%in a single line, where 
%the table name, the name of the columns and
%their type (divided by `,' and () for the data type), and the record offset size would
%be stored (as []). The table metadata should also include which attributes
%are considered as part of the primary key (separate the data type and primary key by using ``;").

For example, consider the following catalog file:

\begin{verbatim}
File: catalog.txt 
------------------------------------
tablename=T
columns=C1:INT,C2:CHAR(5),C3:INT
primary key=C1
recordsize=13
totalsize=65
records=5
tablename=T1
columns=B1:INT,B2:CHAR(255)
primary key=B1
recordsize=259
totalsize=2590
records=10
\end{verbatim}

\subsubsection{Phase 2}

In the second phase, your programming requirements include modifying your application to sort data. This sorting
is the base for the other two important functionalities of the second phase: GROUP BY and BINARY SEARCH.
Sorting should be performed using an in-file sorting approach (a large table will not fit in memory). Therefore, you will have to merge
binary files if you decide to include the merge sort algorithm. You can assume that there will be at most one attribute
in the sorting function.
Once you have coded your ORDER BY operation, you will have to perform a GROUP BY on a previously sorted table. This GROUP BY operation may imply that you have to create temporary tables in order to present the final aggregation. Only consider the SUM aggregation in numerical attributes, and a single attribute in the GROUP BY statement.
The second phase of the project also includes a special operation (that you will not find in a DBMS), which is called
INSERT INTO/ORDER BY. This operation requires you to store the data in sorted fashion. For this statement, only consider a simple SELECT with an ORDER BY.
The last operation that you will have to code for the second phase implies that you will have to modify your SELECT/FROM/WHERE function to allow a faster search in previously sorted tables. Hence, you are required to perform a binary search on a
sorted table.

A summary of the operations (including the submission phase) that you are required to implement are described in Table \ref{tab:Result}.

{\tiny
\begin{table}
	\centering
	\caption{Queries.}
		\begin{tabular}{|l|r|l|p{6.5cm}|}
			\hline
			Operator & Phase				&	Parameters 										& Example\hspace{3cm}\\
			\hline
			\hline
			CREATE TABLE				&1			& Table Name, Columns and Types & CREATE TABLE T ( C1 INT, C2 CHAR(5), C3 INT, PRIMARY KEY(C1));	\\
			INSERT INTO					&1			& Data													& INSERT INTO T VALUES(1,`some text',5); 										\\
			SELECT/FROM					&1			&	Columns												& SELECT C3 FROM T;																					\\
			SELECT/FROM/WHERE		&1			&	Columns ($=,>,<$)							&	SELECT C1 FROM T WHERE C2=`some text';											\\
			SELECT/FROM/JOIN		&1			&	Table Names 									& SELECT T.A FROM T JOIN T1	ON ( T.A = T1.B );								\\
			INSERT INTO 				&1			& Table, Select Query						& INSERT INTO T SELECT B FROM T1;														\\
			SHOW TABLES					&1			& None													& SHOW TABLES;																							\\
			SHOW TABLE					&1			& Table Name										& SHOW TABLE T;\\
			QUIT								&1			& None													& QUIT;\\
			SELECT/FROM/ORDER BY&2			&	Columns, Table, Columns				& SELECT C1, C2 FROM T ORDER BY C1;															\\
			INSERT INTO/ORDER BY&	2		& Table, SELECT ORDER BY Query		& INSERT INTO T3 SELECT T\_A FROM T1 ORDER BY B;									\\
			SELECT/FROM/GROUP BY&	2		&	Columns, Table, Columns				& SELECT C1, SUM(C3) FROM T GROUP BY C1;											\\
			SELECT/FROM/WHERE (Binary search)		&	2		&	Columns, Table, Columns				& SELECT * FROM T3 WHERE C2=`some text'; 			\footnotemark[1]		
\\
			\hline
		\end{tabular}
	\label{tab:Result}
\end{table}
}
 \footnotetext[1]{In a sorted table, you are required to perform a binary search.}


%There limit in the number of columns to include in a table is equivalent
%to the maximum value of an integer.
Some restrictions that you need to take into consideration include:

\begin{itemize}
	\item Do not consider NULLs for this project.
	\item Only two tables are going to be considered in the JOIN statement.
	\item Only one condition will be considered in the WHERE statement.
	\item Return an error message for every invalid operation (e.g. load data into a non-existent table.)
	%\item The CSV files will be consider to contain  only ``," as field separators.
	\item	Do not consider nested queries for this project.
\end{itemize}

\section{Program call and Output}

Here is a specification of input/output.

\begin{itemize}
	\item uhdbms script=$filename$
	\item uhdbms $<query>$
\end{itemize}
\textbf{Example of call:} \\
\\
1) Example 1:
\\
uhdbms CREATE TABLE T ( C1 INT, C2 CHAR(5), C3 INT,PRIMARY KEY(C1));
\\
	Created TABLE T successfully
\\
	$SQL>$ SELECT C1 FROM T
\\
	$SQL>$ quit;
\\
\\
2) Example 2:
uhdbms script=xyz.txt $>output.txt$\footnote{The output.txt file should contain all the results and messages of the SQL queries in the xyz.txt file.}

\end{document}
