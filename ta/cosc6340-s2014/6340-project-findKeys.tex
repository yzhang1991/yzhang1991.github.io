\documentstyle[times,11pt,verbatim,js-singlespace]{article}  % 10pt default?

\title{COSC6340: Database Systems\\
Project: Finding Candidate Keys and Foreign Keys in SQL
}

\author{Instructor: Carlos Ordonez}
\date{}

\begin{document}

\maketitle

\section{Introduction}
%
You will develop a Java program that generates SQL code to find candidate keys and
foreign keys in a database.  % set?

You will use a DBMS supporting SQL.
All keys are assumed to have one column.
The database is simple a set of tables in SQL.


\section{Program requirements}

Your program will generate SQL code based on a list of tables given in a text file. 
Basically, you will have to compute 
projections $\pi$ 
and 
joins $\Join$ between two tables 
in order to ``guess" which columns may be candidate keys and foreign keys.
Remember a foreign key $K$ in $T_1$ is valid 
if all its values exist in some referenced table $T_2$ where $K$ is a PK.
(i.e. all $T_1$ rows with different $K$ values match some row in $T_2$).


Relational queries will be evaluated with SQL. You have freedom to program your SQL queries
(nested, derived tables, views, etc). Cursors are not necessary.
%
If the size of the result table is non-empty (one or more rows) you will assume there
is a foreign key between both tables (or ER relationship).
Remember that joins are computed over columns (attributes) from the same data type (domains).
Columns (attributes) may have, but are not required to have the same name.
That is, the program cannot assume the a potential foreign column has the same name in two tables.
On the other hand, if a column has the same name in two tables it could
be a foreign key in one table, but it cannot be guaranteed.
However, your program needs to assume that both columns have the same data type.
For instance an integer column cannot be joined with a string column.
Your program only needs to handle integers and strings (dates, float not required).

In most cases, there will be a foreign key referring to one table, but there may
database instances where a foreign key may refer to two tables.
In general, a foreign key will have the same name on both related tables, BUT
that is not a requirement. That is, your program must still test every column,
even if the column names are different.
Also, data types must be the same in order to join two tables, but your program
CANNOT assume a column with the same name on two tables has the same data type.



\section{Program call, input and Output}

The program basically takes a text file with the database schema and a threshold for the number
Here is a specification of input/output.
of rows in a successful join.

\begin{itemize}
\item syntax for call:  {\bf ``java FindKeys database=db.txt;minjoinrows={integer$|$all}"}.
\item Notice parameters have a specific name (case insensitive).

\item The {\bf minjoinrows} threshold has default value 1 (if no value provided).
Integer can be some minimum number of rows of the referencing table 
 (for instance size of one table if known).
 "all" means the referencing table rows match with no row excluded
from the referencing table (i.e. the foreign key table).
Zero is not acceptable (think why).
If the parameter is "all" then the foreign key column must not have nulls. 
However, there may be test cases with nulls in a column 
 and thus your program can state that such column is not likely a FK.

\item
Notice parameters are not separated by spaces, but by ;.

\item the output is another text file called "foundKeys.txt", which provides
primary key as "PK",
candidates keys labeled with "CK" and foreign keys with "FK".
You should label "PK" the first candidate key.
It is desirable tables appear in the same order as the input file.
\end{itemize}


\subsection{Programming details}

\begin{enumerate}

\item The user must provide an input text file and an evaluation option parameter (certification or decomposition). %a candidate key and a list of nonkey attributes at a minimum; 
If any of these parameters
are missing, your program should return an error message. 

\item The program will be developed in Java; any version is OK. Your program will be compiled 
by the TA from the command line with "javac". 
You can use any GUI for development (e.g. Eclipse).

\item You need to use JDBC to submit queries, but your program should be able to generate an SQL script file for all the queries used in your program.
The TA will provide the IP address of the DBMS server.
Your program will be tested with tables in this DBMS server.

\item You can include an optional GUI, but output in simple text form is the only requirement. You can get +1 if you include a nice/intuitive GUI (with a README file). 
Although, you can only get this extra credit if your program 
finds foreig keys correctly.

\item 
You can create temporary tables.
Always remember to drop any derived table in advance.

\item This program is a single phase project. 
You can ask any questions by the Google newsgroup (preferably) 
or email (you must have a good reason not to use the Google newsgroup). 

You can also come to the TA or my office hours with specific programming questions
(if the question is not posted or this homework description is unclear).
In addition, we will have short disucssion on the homework every week.

\item Each table will have no more than 10 columns.
The database will have no more than 20 tables.

\item Tables may be empty and input text files may be empty.

\item 8 test cases will be "clean" databases. 2 test cases may contain nulls.
There will be at most 2 test cases with input files with errors 
(table name starts with number, table with no columns, commas missing, etc).
Your program should not crash under input files with errors.


\end{enumerate}


\section{Examples of input and output}


Here are two examples of database schemas. Notice there is one table schema per line.
Notice data type is not specified.
1st example of input file with integer columns:
\begin{verbatim}
 T1(A,B,C)
 T2(B,C,Z)
 T3(C,D,E,F)
\end{verbatim}


Example of output for 1st example (db-easy.txt):
\begin{verbatim}
T1(A(PK),B(FK=T2.B)),C(FK=T3.C))
T2(B(PK),C(FK=T3.C),Z(CK))
T3(C(PK),D,F(CK),E,F)
\end{verbatim}




2nd Example of input file (db-store.txt):
\begin{verbatim}
 Employee(employeeid,empname,empsalary,departmentid)
 Department(deptid,deptname,deptSalesAmt,deptItems)
 Product(SKU,productName,productPrice,deptid)
\end{verbatim}

Output for store database:
\begin{verbatim}
 Employee(employeeid(PK),empname(CK),empsalary,departmentid(FK=department.deptid)
 Department(deptid(PK),deptname,deptSalesAmt,deptItems)
 Product(SKU(PK),productName,productPrice,deptid(FK=DEPARTMENT.deptid))
\end{verbatim}


\section{Turn in project}

One email  with a zip file attached. The subject should clearly specify course and 
project \#.
Do not include class files or Java libraries; only your Java and SQL code.
Include a readme file to compile/run (instructions, notes, errors if any)
and small database example where your program runs.
It is OK to inclue some saample SQL and output generated by your program.
Do not include a large Java library in your zip file.
Your program should work with the existing JDK (1.5, 1.6).
Therefore, your zip file is expected to be **small**.

If you developed a GUI explain how to run it in the README file. The GUI is at most +10\%
and will count only if your program passes most test cases.
Also, your program should work from the command-line without the GUI.

Your project must be sent by 10pm on the due date.
No late submissions are accepted, unless there is a previous discussion 
and acceptable justification with the professor (the TA cannot make exceptions).

Example:
email with subject "COSC6340: Project1Team01" (two digits for team number). 
The zip file should be called "Project1Team01.zip" 
and must contain all .java files and the README.TXT file.



\end{document}
